%-------------------------------------------------------------------------------
%	SECTION TITLE
%-------------------------------------------------------------------------------
\vspace{4mm}
\cvsection{경력}


%-------------------------------------------------------------------------------
%	CONTENT
%-------------------------------------------------------------------------------
\begin{cventries}

%---------------------------------------------------------

  \cventry
    {소프트웨어 엔지니어} % Job title
    {Moreh Corp.} % Organization
    {서울, 대한민국} % Location
    {2021년 03월 - 현재} % Date(s)
    {
      \begin{cvitems}
        \item {다중 노드 GPU 클러스터 환경에서 컨텍스트 병렬화(Ulysses + Ring Attention, zigzag 로드 밸런싱)를 활용한 long context LLM prefill 전용 엔진 개발. gpt-oss-120b 모델의 32K/64K 컨텍스트에서 TTFT 1초 미만 달성 및 이기종 클러스터 내 레거시 GPU 활용 지원. Attention Sink를 적용한 컨텍스트 윈도우 어텐션용 FlashAttention 스타일 Triton 커널 구현.}
        \item {사용자 수준 Python 라이브러리(PyTorch 등)부터 GPU 커널까지 아우르는 풀스택 분산 머신러닝 프레임워크 구현. 연산 그래프 중간 표현(IR), 실행 시간 예측을 위한 그래프 수준 비용 모델, 이를 활용해 최적 실행 계획을 생성하는 컴파일러 구현.}
        \item {모델 및 데이터 전송 루틴 전반에 걸쳐 성능 병목 구간 프로파일링 및 최적화 수행.}
        \item {학습 이상 현상 진단 및 수정을 통해 모델 안정성 및 성능 개선.}
      \end{cvitems}
    }

%---------------------------------------------------------
\vspace{4mm}
  \cventry
    {소프트웨어 엔지니어} % Job title
    {\mycompany{}. Co., Ltd.} % Organization
    {경기도, 대한민국} % Location
    {2019년 02월 - 2021년 03월} % Date(s)
    {
      \begin{cvitems}
        \item {Linux 환경에서 Windows 바이너리 실행을 위한 Windows API 호환 계층 구현 (\mycompany{}: Linux 기반 운영체제).}
	\item {런타임 x86 기계어 코드 생성 및 인라인 어셈블리를 활용해 Windows/Linux 호출 규약을 연결하는 Windows 프로세스 간 COM 호출 프록시 함수 설계 및 개발.}
      \end{cvitems}
    }

%---------------------------------------------------------
\end{cventries}
\vspace{4mm}
